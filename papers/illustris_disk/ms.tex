
\documentclass[useAMS,usenatbib]{mn2e}
%\documentclass[usenatbib]{mn2e}
\usepackage{graphicx}
%\usepackage[nottoc]{tocbibind}
\setlength{\topmargin}{-1.2cm}

% If your system does not have the AMS fonts version 2.0 installed, then
% remove the useAMS option.
%
% useAMS allows you to obtain upright Greek characters.
% e.g. \umu, \upi etc.  See the section on "Upright Greek characters" in
% this guide for further information.
%
% If you are using AMS 2.0 fonts, bold math letters/symbols are available
% at a larger range of sizes for NFSS release 1 and 2 (using \boldmath or
% preferably \bmath).
%
% The usenatbib command allows the use of Patrick Daly's natbib.sty for
% cross-referencing.
%
% If you wish to typeset the paper in Times font (if you do not have the
% PostScript Type 1 Computer Modern fonts you will need to do this to get
% smoother fonts in a PDF file) then uncomment the next line
% \usepackage{Times}

%%%%% AUTHORS - PLACE YOUR OWN MACROS HERE %%%%%

% Journal definitions
\newcommand{\apj}[1]{ApJ}
\newcommand{\mnras}[1]{MNRAS}
\newcommand{\apjl}[1]{ApJL}
\newcommand{\pasj}[1]{PASJ}
\newcommand{\aj}[1]{AJ}
\newcommand{\nat}[1]{Nature}
\newcommand{\aap}{A\&A}

%%%%%%%%%%%%%%%%%%%%%%%%%%%%%%%%%%%%%%%%%%%%%%%%

\title[Flux ratio anomalies from disks]{
%Gravitational Lens Modelling of flux-ratio Anomalous System B1555+375
SHARP IV: Flux ratio anomalies from disks in the Illustris simulation
}
\author[Hsueh et al.]{Jen-Wei Hsueh$^{1}$\thanks{E-mail:
jwhsueh@ucdavis.edu (JWH); otheremail@otheraddress (ANO)} and A. N.
Other$^{2}$\\
$^{1}$Physics Dept., University of California, Davis, 1 Shields Ave.
Davis, CA 95616, USA\\
$^{2}$Building, Institute, Street Address, City, Code, Country}
\begin{document}

%\date{Accepted 1988 December 15. Received 1988 December 14; in original form 1988 October 11}

\pagerange{\pageref{firstpage}--\pageref{lastpage}} \pubyear{2015}

\maketitle

\label{firstpage}

\begin{abstract}

\end{abstract}

\begin{keywords}
gravitational lensing
\end{keywords}

\section{Introduction}
\section{The Illustris simulation}
\section{Disks selection}
\begin{itemize}
\item discuss how realistic disks in illustris are
\end{itemize}
Resolution: stellar mass cut between 7E9Msun~1E10Msun. Due to the resolution limitation, our disk galaxy samples are from the high-mass range of disk galaxy population, with mass range [].  

\subsection{Morphological selection}
Photometric properties are commonly used in surveys to identify the morphology type of galaxies. Star particles in simulation can also be visualized (SUNRISE) and provide photometric observants for individual galaxy. We apply the following two criteria to identity a disk galaxy in Illustris simulation: 1) The galaxy has a 
better fit (i.e. a smaller $\chi^2$) in exponential disk light profile than in de Vaucouleur light profile, and 2) the galaxy has a Sersic index smaller than two. The sample selected by these photometric properties are called morphological selection in this work.

\subsection{Kinematical selection I}
In kinematical view, stellar disk is a group of stars co-rotate in the same direction with the common symmetric axis. One of the advantages in simulations is that one can access to the full detail of kinematical properties for single mass particle. These kinematical properties provide an approach to decompose different components in disk galaxy: thin disk, thick disk, and spheroid.
z-component angular momentum (symmetric axis)
Jz vs binding energy: $J_{circ}$ curve
circularity parameter $\epsilon_z = Jz/J_{circ}$

decomposition: thin disk+thick disk+spheroid

thin disk star fraction larger than 40\%

\subsection{Kinematical selection II}
stars in spheroid: bulge stars
bulge star fraction less than 60\%
\section{Disk properties}
\begin{itemize}
\item how many galaxies have disks
\item mass in disk
\item redshift evolution
\item comparison with B1555 and SWELLS
\item observability of the disks
\end{itemize}
\section{Flux ratio anomalies from disks}

\section{Discussion}
\section*{Acknowledgments}

\bibliographystyle{mn2e}
\bibliography{reference.bib}




\label{lastpage}

\end{document}
