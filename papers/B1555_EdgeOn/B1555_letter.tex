\documentclass[manuscript]{emulateapj}

\shorttitle{Gravitational Lens Modeling of Flux Ratio Anomalous System B1555+375}
\shortauthors{TBD}

\begin{document}

\title{Gravitational Lens Modeling of Flux Ratio Anomalous System B1555+375}
\author{TBD}

\begin{abstract}
With the latest infrared image of B1555+375, an edge-on disk is shown to over cross the strong lensed images. Indicating the flux ratio anomalie of B1555+375 is dominanted by its edge-on disk structure. We present a lens model with SIE bulge and exponential disk, base on the radio observation from \citet{Marlow}. And apply this model to SHARP infrared image.

\end{abstract}

\keywords{ gravitaional lensing}

\section{Introduction}
To understand substructions and mass distribution within dark matter haloes, observations in Local Group bring the first view of mass profile on small scale by searching satellite galaxies directly. However, concordant cold dark matter cosmology predicts more substructures in numerical simulations than in observation results around our Milky Way. This discrepency arises the famous ``missing satellite'' problem. Searching more satellite galaxies beyond Local Group is definite one of the keys to solve this discrepency, though, in approaches with huge differences from people used to do in Local Group. Most satellite galaxies are faint, which are extremely hard to directly detect in the system beyond Milky Way. Futhermore, some low-mass dark matter subhaloes are considered to be too small to contain dwarf galaxies. These factors make gravitational effects be powerful and promising tools in the substructure searching.\\   
Flux ratio anomalies are seen in strong lensing systems which have multiple images that are spacially close to each other. These ``merging doubles'' share the same surface brightness in principle but happend to have different surface brightness due to perturbations in gravitational potential. Little perturbation from substructures (mass range?) is sufficent to bring flux ratio anomalies in strong lensed systems. In the observational view, radio observation is the best way to probe these flux ratio anomalous systems because it doesn't have severe extinction issue like in optical or infrared band. Also, most radio lensed images are point sources which are easier to model. \\
On the other hand, results from numerical simulation haven't reach a concise conclusion on the cause of flux ratio anomaly. There are other possiblities can give rise to flux ratio anomaly. Before gathering any satistics or serious analysis, we must fully consider those non-substructure effects. Otherwise it could affect the estimation of substructure mass and even give false signal when searching possible satellite candidates. Among those non-substructure effects, we are interested in the gravitational effect caused by structure of lens galaxy itself. Thus, infrared image with high spatial resolution is essential in our study.\\
The Strong lensing at High Angular Resolution Program(SHARP) is a project aims on known quadruple and Einstein ring lesing systems. We obtain high-resoltion images from Keck adaptive optics(AO) Hubble Space Telescope(HST) observations. In the latest SHARP imaging, we found clear edge-on disk within some known flux ratio anomalous lenses. We model one system, B1555+375, with a center bulge and an edge-on disk, intending to probe the possibility of that structure from lens galaxy can cause flux ratio anomaly. We present the latest SHARP data in part 2, lens model of B1555 in part 3, and discussion in part 4. 

\section{Data Collection \& Reduction}

\section{Lens Modeling}
To model B1555+375, we use the lens modeling code $glafic$ \citep{Oguri} with known position and flux data from the radio observation done by \citet{Marlow}. Morlow et al. first published the flux ratio anomaly detection and lens model of B1555+375. However, their SIE lens model cannot explain the flux ratio anomaly between image A and B (see Fig.6, Table 2 \& 3 in \citet{Marlow}). In K-band image (Fig. 1, left), a recognizable edge-on disk is lie in the center with position angle $~10$ degree. With this information, we model B1555+375 with SIE for bulge and exponential disk model for edge-on disk. The full best-fit parameters are shown in Table 1. We also try on another model with two SIE mass profiles. This model will be used in the following up study with Simona...

The comparison of radio observation and our modeling result is in Figure 2, Table 2 \& 3.\\




\begin{figure}
\plotone{B1555_Kp.eps}
\plottwo{B1555_model.eps}{B1555_residue.eps}
\caption{Left: SHARP K-band image of B1555+375 Middle: Lens model image generated by glafic Right: Residue image of B1555+375.\\
Exponential disk is marked by the green ellipse in residue image. We believe strong dust extinction from the edge-on disk makes B and D no detection in infrared. Thus leaving two negative holes in our residue.\label{fig1}}
\end{figure}

\begin{figure}
\plotone{point_source.eps}
\caption{Radio observation(red open circle) and model-predicted(blue plus sign) image positions of B1555+375. The position of the source is at $(-0.2066,-0.1634)$, marked by a black filled circle. From left to right, four components are A, B (merging doubles), D(lowest spot), and C(right).\label{fig2}}
\end{figure}

%%table1
\begin{table}
\begin{center}
\caption{Lens Model Best-Fit Parameters.\label{tbl-1}}
\begin{tabular}{lcccccc}
%%\tableline\tableline
Model &		&		 &  \\
\tableline\tableline
SIE 	& $x$		& $y$ 		& $\sigma (km/s)$ 	& $e$	& $\theta$  \\  
		&$-0.1883$ 	&$-0.1923$	&$93.74$ 			&$0.27$	& $105.5 \degr$   \\
\tableline
Expdisk	& $x$		& $y$		& $M_{tot} (h^{-1}M_{\odot})$	& $e$	& $\theta$	& $r_e$ \\
		&$-0.1615$  &$-0.2502$	&$1.19\times 10^{10} $ 			&$0.84$ &$7.4 \degr$& $0.20 ''$  \\  
 
\tableline
\end{tabular}
%% Any table notes must follow the \end{tabular} command.
\tablecomments{Positions are offsets from component A measured in unit of arcsec. $e$ is ellipticity and $\theta$ is the position angle measured east of north.}
\end{center}
\end{table}

%%table2
\begin{table}
\begin{center}
\caption{Radio Observation and Model-predicted Image Positions.\label{tbl-2}}
\begin{tabular}{lccccc}
\tableline\tableline
					&Radio	&		 & Model-predicted \\
Component &East &North &East 		&North\\ 
\tableline
A ........ &$0$    		&$0$		&$-0.0000$ &$+0.0000$   \\  
B ........ &$-0.0726$ 	&$+0.0480$	&$-0.0726$ &$+0.0479$   \\  
C ........ &$-0.4117$  &$-0.0280$	&$-0.4117$ &$-0.0280$   \\  
D ........ &$-0.1619$  &$-0.3680$	&$-0.1609$ &$-0.3678$   \\  
\tableline
\end{tabular}
%% Any table notes must follow the \end{tabular} command.
\tablecomments{Radio observation data quoted from Table 2 in \citet{Marlow}. Position offsets are in unit of arcsec.}
\end{center}
\end{table}

%%table3
\begin{table}
\begin{center}
\caption{Flux Ratios of B1555+375 Components.\label{tbl-3}}
\begin{tabular}{lllccc}
\tableline\tableline
				&Radio &Infrared  &Model-predicted\\
\tableline
A/B			&$1.8$ &$3.5$ &$1.7$  \\ 
A/C 		&$2.0$ &$1.0$ &$2.2$  \\
A/D			&$13.0$ &$5.4$ &$8.0$  \\
\tableline
\end{tabular}
%% Any table notes must follow the \end{tabular} command.
\tablecomments{Radio observation data quoted from Table 2 in \citet{Marlow}. Flux ratio between the merging double is much higher in infrared than in radio, which is believed to be caused by strong dust extinction from the edge-on disk.}
\end{center}
\end{table}



We then use an extended source to model SHARP infrared image. In Figure 1, B and D have no detection in K-band, which believed due to strong dust extinction from the edge-on disk. The region and orientation of exponential disk in lens model are consistent with this hypothesis. The flux ratios in infrared are much higher than in radio also supports as an evidence of dust extinction.

\section{Discussion}
We have seen a similar flux ratio anomaly pattern in our edge-on disk lens model of B1555+375. The model predicted point source images also well matched with radio observation. Our modeling result indicates structures of lens galaxy itself can dominant flux ratio anomalies on lensing images. 
Our follow up will be explore more similar SHARP systems with flux ratio anomaly and a recognizable edge-on disk. As shown in Figure 3, the flux ratio anomalies of these systems may also able to be explained by an edge-on disk structure.\\
We still have great expectation on the substructure searching approach from flux ratio anomaly by using strong lensing systems. At the meanwhile, we would like to draw some attention on other factors can cause flux ratio anomaly. These factors also give a clue of the structure and mass distribution of lens galaxy, which can be important to research required really accurate measurements, like time-delay.  For substructure searching, one should reconsider and rule out the possibility of edge-on disk before doing any further analysis.\\
Our result also shows that multi-wavelength image is helpful on exploring flux ratio anomalous systems.

\bibliographystyle{apj}
\bibliography{reference.bib}


%\begin{thebibliography}{}
%\bibitem[Oguri(2010)]{Oguri} Oguri, M. 2010, \pasj, 62, 1017
%\bibitem[Marlow et al.(1999)]{Marlow} Marlow, D. R., et al. 1999, \apj, 118, 654
%\end{thebibliography}

\end{document}

