\documentclass[manuscript]{emulateapj}

\usepackage{color}
\definecolor{simonacolor}{cmyk}{0, 0.9808, 0.4429, 0.1412}
\newcommand{\simona}[1]{\textcolor{simonacolor}{#1}}


\shorttitle{Gravitational Lens Modelling of Flux Ratio Anomalous System B1555+375}
\shortauthors{TBD}

\begin{document}

\title{Gravitational Lens Modelling of Flux Ratio Anomalous System B1555+375}
\author{TBD}

\begin{abstract}
With the latest infrared image of B1555+375, an edge-on disk is shown to over cross the strong lensed images. Indicating the flux ratio anomalie of B1555+375 is dominanted by its edge-on disk structure. We present a lens model with SIE bulge and exponential disk, base on the radio observation from \citet{Marlow}. And apply this model to SHARP infrared image.

\end{abstract}

\keywords{ gravitaional lensing}

\section{Introduction}

A common feature of simulations of structure formation is the presence
of thousands of subhaloes that are associated with larger mass haloes.
However, observations of the Local Group find many fewer satellite
galaxies than are predicted by the simulations, even when taking into
account completeness corrections arising from the limited sky coverage
imposed by the Galactic Plane.  This is the famous ``missing
satellite'' problem \citep{Klypin1999, Moore1999, S07}. In order to understand
this discrepancy, it is imperative to build up a large sample of satellites/subhaloes in galaxies outside the Local Group.  
This process is challenging
due to the faintness of the satellite galaxies.  Furthermore, lower-mass
satellites may not be able to retain the gas needed for ongoing star
formation \citep[e.g.,][]{P11}, rendering them effectively dark.
These factors make gravitational \simona{lensing a powerful and promising tool for the detection of substructure.}

\simona{It was first suggested by \citet{Mao1998} that flux ratio anomalies observed in radio loud multiple images of lensed quasar could be interpreted as the telltale of the presence of substructure in lens galaxies.}
%Flux ratio anomalies are seen in strong lensing systems which have multiple images that are spacially close to each other. These ``merging doubles'' share the same surface brightness in principle but happend to have different surface brightness. 
\simona{Indeed,} little perturbations in the gravitational potential are sufficent to bring flux ratio anomalies in strong lensed systems. 
\simona{Moreover, unlike in the optical or infrared,} in the radio band, lensed images are not sensitive to micro lensing from stars \simona{nor dust extinction}. 
This makes flux ratio anomalies a promising tool to detect dark matter substructures \citep{Dalal2002, N13}.  
\simona{At present, the amount of substructure derived from flux ratio anomalies is in disagreement with predictions from numerical simulations, with the former requiring a larger fraction of mass in substructure than expected \citet{Xu14}. While this could be the result of small sample statistics, it could also be the indication that flux ratio anomalies have a different origin than substructure.} \simona{An alternative method for the detection of substructure in gravitational lens galaxies based on the surface brightness distribution of extended arcs and Einstein ring has been introduced by \citet{K05,V09}. Interestingly, the fraction of substructure measured with this \emph{gravitational imaging technique} is currently in agreement with theoretical expectations \citep{V14a}.} 

While future large-scale surveys, such as the LSST, \simona{are expected to deliver new large samples of gravitationally lensed quasar, in the near future the discovery of new systems using} narrow-line quasar emission \citep{N14} \simona{will provide new invaluable statistical perspective on current constraints based on flux ratio anomalies. 
In this paper, we explore the alternative idea that flux ratio anomalies are not related to the presence of substructure but originate in a more complex host galaxy mass distribution than initially considered. To this end we use new infra-red data from the Strong lensing at High Angular Resolution Program(SHARP). The main goal of the SHARP} project \simona{is to re-observe} known quadruple and Einstein ring lens systems at a much higher angular resolution by making use of the Keck adaptive optics (AO) telescope and large interferometric arrays \citep{SHARP12,V12} . In particular, to better understand the cause of flux ratio anomalies in strong lens systems \simona{we focus on the lens modelling of the lens system B1555+375 (included in both the observational and theoretical analysis of  \citet{Dalal2002} and \citet{Xu14}, respectively) by making use of} the new information obtained from the latest SHARP infrared imaging. \simona{The paper is organized as follows: the latest SHARP data is introduced in section 2; in section 3 we present the lens model while the results of this analysis and their implication are discussed in section 4.}
%However, only small number of flux ratio anomalous systems in \simona{the} radio are \simona{currently} avaliable. Besides waiting for future large-scale survey such as LSST to find thousands of more new lens galaxies, another approach \simona{is to} seek out more flux ratio anomalous systems \simona{using} narrow-line quasar emission \citep{N14}. 
%	With expected growth in the number of samples, several groups have developed the technique to estimate the properties of substructures from flux ratio anomalous systems. \citet{V09, H13} both analyze the strong lens imaging to detect substructures lie infront of Einstein ring. Their methods improve the detectable mass limit and also show that high spatial resolution imaging is essential for the substructure study. On the other hand, numerical simulations on structure formation do not reach the concise conclusion on flux ratio anomalous systems. The latest study on the cause of flux anomalies from \citet{Xu14} shows the low probability of the existence of substructures in their studied systems. 
%	The Strong lensing at High Angular Resolution Program(SHARP) is a project aims on known quadruple and Einstein ring lensing systems \citep{SHARP12}. We obtain high-resoltion images from Keck adaptive optics(AO) and Hubble Space Telescope(HST) observations. 
%To better understand the cause of flux ratio anomaly in strong lensing systems, we model B1555+375, one of the systems discussed in \citet{Xu14}, with the new information obtained from the latest SHARP infrared imaging. We present the latest SHARP data in part 2, lens model of B1555+375 in part 3, and discussion in part 4. 

\section{Data Collection \& Reduction}

\section{Lens Modelling}
To model B1555+375, we use the lens modelling code {\tt glafic}
\citep{Oguri}.  The inputs to the model are the observed image positions
and flux densities measured by the radio observations of \citet{Marlow}.
These authors present a singular isothermal ellipsoid (SIE) model that
provides the first fit to the image positions but
does not reproduce the strong flux ratio anomaly between components A and B.
(see Fig.6, Table 2 \& 3 in \citet{Marlow}). In K-band
image (Fig. 1, left), a recognizable edge-on disk is lie in the center
with position angle $~10$ degree. With this information, we model
B1555+375 with two models. The first model consists of an SIE bulge
and an exponential edge-on disk (SIE+Expdisk). The second model
consists of two SIE mass profiles (2-SIE), for bulge and disk
respectively. We assume that the lens galaxy has $z_{l}=0.5$ and the source has $z_s=1.5$. The full best-fit parameters are shown in Table 1. Both
of these models predict fairly matched results with the radio
observation. The comparison of observation and SIE+Expdisk result is
in Figure 2, Table 2 \& 3.
%These authors present a singular isothermal ellipsoid (SIE) model that
%provides an excellent fit to the image positions (?? IS THIS TRUE?--NO we get a better model result)



%\begin{figure}
%\plotone{B1555_Kp.eps}
%\plottwo{B1555_model.eps}{B1555_residue.eps}
%\caption{Left: SHARP K-band image of B1555+375 Middle: Lens model image generated by glafic Right: Residue image of B1555+375.\\
%An edge-on disk can be seen in the center of SHARP K-band image. Exponential disk is marked by the green ellipse in residue image. We believe strong dust extinction from the edge-on disk makes B and D no detection in infrared. Thus leaving two negative holes in our residue.\label{fig1}}
%\end{figure}
%
%\begin{figure}
%\plotone{point_source.eps}
%\caption{Radio observation(red open circle) and model-predicted(blue plus sign) image positions of B1555+375. The position of the source is at $(-0.2066,-0.1634)$ for SIE+Expdisk model, marked by a black filled circle. From left to right, four components are A, B (merging doubles), D(lowest spot), and C(right).\label{fig2}}
%\end{figure}

%%table1
\begin{table}
\begin{center}
\caption{Lens Model Best-Fit Parameters.\label{tbl-1}}
\begin{tabular}{ccccccc}
%%\tableline\tableline
		&SIE+Expdisk& 2-SIE		   
\\
\tableline\tableline
$x_1$  	& $-0.1883$	& $-0.1809$	  \\
$y_1$	&$-0.1923$	&$-0.1702$	  \\
$x_2$	&$-0.1615$ 	&$-0.1592$	  \\
$y_2$	&$-0.2502$	& $-0.1833$	  \\
$\sigma_1$	&$92.90$ &	$95.05$	  \\
$M_{tot} / \sigma_2$& $1.17\times 10^{10} $  &$81.36$ 	 \\  
$e_1$	& $0.27$	& $0.27$ \\  
$\theta_1$	&$105.5 \degr$ & $97.7 \degr$	 \\
$e_2$	&$0.84$	&$0.83$      \\
$\theta_2$	&$7.4\degr$ &$7.4\degr$  \\
$r_e$	& $0.20 ''$ &  -- \\
\tableline

\end{tabular}
%% Any table notes must follow the \end{tabular} command.
\tablecomments{Subscripts 1 and 2 represent bulge and disk components in each model respectively. Positions are offsets from radio image component A measured in units of arcsec. The velocity dispersion $\sigma$ is in units of $km/s$ and disk mass $M_{tot}$ is in units of $h^{-1} M_{\odot}$. $e$ is ellipticity and $\theta$ is the position angle measured east of north.}
\end{center}
\end{table}

%%table2
\begin{table}
\begin{center}
\caption{Radio Observation and Model-predicted Image Positions.\label{tbl-2}}
\begin{tabular}{lccccc}
\tableline\tableline
					&Radio	&		 & Model-predicted \\
Component &East &North &East 		&North\\ 
\tableline
A ........ &$0$    		&$0$		&$-0.0000$ &$+0.0000$   \\  
B ........ &$-0.0726$ 	&$+0.0480$	&$-0.0726$ &$+0.0479$   \\  
C ........ &$-0.4117$  &$-0.0280$	&$-0.4117$ &$-0.0280$   \\  
D ........ &$-0.1619$  &$-0.3680$	&$-0.1609$ &$-0.3678$   \\  
\tableline
\end{tabular}
%% Any table notes must follow the \end{tabular} command.
\tablecomments{Radio observation data quoted from Table 2 in \citet{Marlow}. Position offsets are in unit of arcsec.}
\end{center}
\end{table}

%%table3
\begin{table}
\begin{center}
\caption{Flux Ratios of B1555+375 Components.\label{tbl-3}}
\begin{tabular}{lccccc}
\tableline\tableline
				&Radio &Infrared  &Model-predicted\\
\tableline
A/B			&$1.8$ &$3.5$ &$1.7$  \\ 
A/C 		&$2.0$ &$1.0$ &$2.2$  \\
A/D			&$13.0$ &$5.4$ &$8.0$  \\
\tableline
\end{tabular}
%% Any table notes must follow the \end{tabular} command.
\tablecomments{Radio observation data quoted from Table 2 in \citet{Marlow}. Flux ratio between the merging double is much higher in infrared than in radio, which is believed to be caused by strong dust extinction from the edge-on disk.}
\end{center}
\end{table}



We then use an extended source to model SHARP infrared image. In Figure 1, B and D have no detection in K-band, which believed due to strong dust extinction from the edge-on disk. The region and orientation of exponential disk in lens model are consistent with this hypothesis. The flux ratios in infrared are much higher than in radio also supports as an evidence of dust extinction.

\section{Discussion}
The abundance of substructure within galaxy-size halo has been an undetermined problem in structure formation studies. The discrepency between numerical simulations and observations within the Local Universe gives two possible arguments: an alternative structure formation model or the selection bias of local universe samples. Lensing shows a promising potential on substructure detection beyond the Local Group. The main reason is that the gravitational perturbation from the substructure reflects the total mass, regardless it's luminous or not.

Base on the fact that surface brightness is sensitive to perturbations in gravitational potential, flux ratio anomalies has been shown as one of the most effective methods to detect substructures \citep{Dalal2002} (more ref). Until now there are a dozen(?) known flux ratio anomalous systems and the number must increase in the future survey. Radio data provides reliable positions and fluxes of source for constraint and modelling because it avoids the problem of severe extinction and micro lensing from stars. On the other hand, optical or near infrared imaging also gives useful information about the lensing system. In the previous paper of SHARP, \citet{SHARP12} show that high resolution IR imaging improves the lens modelling. The lack of knowledge about lens galaxy, e.g. modelling with radio data alnog, may lead to an improper lens model.

The idea of detecting substructures with flux ratio anomalies assumes that lens galaxy has completely smooth gravitational potential. Thus the perturbation is considered to be from faint satellite galaxy or ``dark'' subhalo. Numerical simulations have examined this approach with unkown systems. In \citet{Xu14}, only 1 out of 6 lens systems can be reproduced the flux anomaly by adding subhalos. The rest of them are only few percent likely or very unlikely to have their flux anomalies due to substructure, including B1555+375. Their results support that even using radio data, there are other possible reasons could cause flux anomalies. The answer to this concern may be explained by the high resolution IR imaging. In the SHARP K-band imaging of B1555+375, an edge-on disk lies across between the positions of radio merging double A \& B.

In sec. 2, we present new lens models of B1555+375 with a bulge and an edge-on disk. Our models better match to the radio image positions than in \citet{Marlow} and also explain the flux ratio anomaly between components A and B. In our modelling, substructure is not neccessary to exist to cause the observed flux ratio anomaly. This indicates the flux ratio anomaly in B1555+375 is resulted from the structure of lens galaxy itself, which is the edge-on disk seen in the SHARP infrared imaging. Our result is consistent with the conclusion from \citet{Xu14}. Their simulation gives a low probability on the hypothesis that substructure causing flux anomaly in B1555+375.

With our discover in B155+375, we show that the lens galaxy structure can dominate the flux ratio anomalous phenomenon in strong lensing system. This leads to a more general reminder. Not all of the flux anomalies indicate the existence of substructures. One must fully consider non-substructure effects before gathering any statistics or serious analysis. Otherwise, these ``false'' substructures that detected by flux anomalies can serverly biased the data and analysis. For the flux ratio anomalies caused by real dark matter substructures and dwarf galaxies, non-stucture effects would also influence the estimated mass and other derived properties. Thus, we would like to draw some attention on these non-substructure effects. Once we understand more about their role in flux anomalous phenomenon, we can be more confident on searching unseen dwarf galaxies and dark matter substructures by searching flux ratio anomalous systems.

Our result also shows that multi-wavelength image is helpful on exploring flux ratio anomalous systems. The radio and narrow band observations are the best approaches to obtain accurate flux ratio data. However, optical and infrared imaging can provide more information on the structure of lens galaxy. With the first system being analyzed and shown that an edge-on disk can cause flux ratio anomaly, our follow up will be explore other SHARP systems. In Figure 3, all of these systems have a recognizable edge-on disk in infrared imaging. The flux ratio anomalies of these systems may also able to be explained by an edge-on disk structure.\\
 



\bibliographystyle{apj}
\bibliography{reference.bib}


%\begin{thebibliography}{}
%\bibitem[Oguri(2010)]{Oguri} Oguri, M. 2010, \pasj, 62, 1017
%\bibitem[Marlow et al.(1999)]{Marlow} Marlow, D. R., et al. 1999, \apj, 118, 654
%\end{thebibliography}

\end{document}

