\documentclass[manuscript]{emulateapj}

\usepackage{color}
\definecolor{simonacolor}{cmyk}{0, 0.9808, 0.4429, 0.1412}
\newcommand{\simona}[1]{\textcolor{simonacolor}{#1}}


\shorttitle{Gravitational Lens Modelling of Flux Ratio Anomalous System B1555+375}
\shortauthors{TBD}

\begin{document}

\title{Gravitational Lens Modelling of Flux Ratio Anomalous System B1555+375}
\author{TBD}

\begin{abstract}
With the latest infrared image of B1555+375, an edge-on disk is shown to over cross the strong lensed images. Indicating the flux ratio anomalie of B1555+375 is dominanted by its edge-on disk structure. We present a lens model with SIE bulge and exponential disk, base on the radio observation from \citet{Marlow}. And apply this model to SHARP infrared image.

\end{abstract}

\keywords{ gravitaional lensing}

\section{Introduction}

A common feature of simulations of structure formation is the presence
of thousands of subhaloes that are associated with larger mass haloes.
However, observations of the Local Group find many fewer satellite
galaxies than are predicted by the simulations, even when taking into
account completeness corrections arising from the limited sky coverage
imposed by the Galactic Plane.  This is the famous ``missing
satellite'' problem \citep{Klypin1999, Moore1999, S07}. In order to understand
this discrepancy, it is imperative to build up a large sample of satellites/subhaloes in galaxies outside the Local Group.  
This process is challenging
due to the faintness of the satellite galaxies.  Furthermore, lower-mass
satellites may not be able to retain the gas needed for ongoing star
formation \citep[e.g.,][]{P11}, rendering them effectively dark.
These factors make gravitational \simona{lensing a powerful and promising tool for the detection of substructure.}

\simona{It was first suggested by \citet{Mao1998} that flux ratio anomalies observed in radio loud multiple images of lensed quasar could be interpreted as the telltale of the presence of substructure in lens galaxies.}
%Flux ratio anomalies are seen in strong lensing systems which have multiple images that are spacially close to each other. These ``merging doubles'' share the same surface brightness in principle but happend to have different surface brightness. 
\simona{Indeed,} little perturbations in the gravitational potential are sufficient to bring flux ratio anomalies in strong lensed systems. 
\simona{Moreover, unlike in the optical or infrared,} in the radio band, lensed images are not sensitive to micro lensing from stars \simona{nor dust extinction}. 
This makes flux ratio anomalies a promising tool to detect dark matter substructures \citep{Dalal2002, N13}.  
\simona{At present, the amount of substructure derived from flux ratio anomalies is in disagreement with predictions from numerical simulations, with the former requiring a larger fraction of mass in substructure than expected \citet{Xu14}. While this could be the result of small sample statistics, it could also be the indication that flux ratio anomalies have a different origin than substructure.} \simona{An alternative method for the detection of substructure in gravitational lens galaxies based on the surface brightness distribution of extended arcs and Einstein ring has been introduced by \citet{K05,V09}. Interestingly, the fraction of substructure measured with this \emph{gravitational imaging technique} is currently in agreement with theoretical expectations \citep{V14a}.} 

While future large-scale surveys, such as the LSST, \simona{are expected to deliver new large samples of gravitationally lensed quasar, in the near future the discovery of new systems using} narrow-line quasar emission \citep{N14} \simona{will provide new invaluable statistical perspective on current constraints based on flux ratio anomalies. 
In this paper, we explore the alternative idea that flux ratio anomalies are not related to the presence of substructure but originate in a more complex lens galaxy mass distribution than initially considered. To this end we use new infra-red data from the Strong lensing at High Angular Resolution Program(SHARP). The main goal of the SHARP} project \simona{is to re-observe} known quadruple and Einstein ring lens systems at a much higher angular resolution by making use of the Keck adaptive optics (AO) telescope and large interferometric arrays \citep{SHARP12,V12} . In particular, to better understand the cause of flux ratio anomalies in strong lens systems \simona{we focus on the lens modelling of the lens system B1555+375 (included in both the observational and theoretical analysis of  \citet{Dalal2002} and \citet{Xu14}, respectively) by making use of} the new information obtained from the latest SHARP infrared imaging. \simona{The paper is organized as follows: the latest SHARP data is introduced in section 2; in section 3 we present the lens model while the results of this analysis and their implication are discussed in section 4.}
%However, only small number of flux ratio anomalous systems in \simona{the} radio are \simona{currently} avaliable. Besides waiting for future large-scale survey such as LSST to find thousands of more new lens galaxies, another approach \simona{is to} seek out more flux ratio anomalous systems \simona{using} narrow-line quasar emission \citep{N14}. 
%	With expected growth in the number of samples, several groups have developed the technique to estimate the properties of substructures from flux ratio anomalous systems. \citet{V09, H13} both analyze the strong lens imaging to detect substructures lie infront of Einstein ring. Their methods improve the detectable mass limit and also show that high spatial resolution imaging is essential for the substructure study. On the other hand, numerical simulations on structure formation do not reach the concise conclusion on flux ratio anomalous systems. The latest study on the cause of flux anomalies from \citet{Xu14} shows the low probability of the existence of substructures in their studied systems. 
%	The Strong lensing at High Angular Resolution Program(SHARP) is a project aims on known quadruple and Einstein ring lensing systems \citep{SHARP12}. We obtain high-resoltion images from Keck adaptive optics(AO) and Hubble Space Telescope(HST) observations. 
%To better understand the cause of flux ratio anomaly in strong lensing systems, we model B1555+375, one of the systems discussed in \citet{Xu14}, with the new information obtained from the latest SHARP infrared imaging. We present the latest SHARP data in part 2, lens model of B1555+375 in part 3, and discussion in part 4. 

\section{Data Collection \& Reduction}

\subsection{Hubble Space Telescope Archival Imaging}

\subsection{Keck Adaptive Optics Imaging}

The B1555+375 system was observed using the NIRC2 camera on the Keck 2
Telescope on the night of 2012 May 16 UT.  The adaptive optics system
was used, with the corrections derived from the laser guide star and a
$R$=14.4 tip-tilt that was located 45$^{\prime\prime}$ from the lens
system.  The ``narrow camera'' mode was used, giving a field of view
of roughly 10$^{\prime\prime}$\ on a side and a pixel scale of 10~mas.
Six dithered 300~s exposures were obtained in $K^{\prime}$ band.  The
data were reduced with the standard SHARP pipeline, which is a
python-based package that is a refinement of the process described in
\citet{Auger_EELS1}.  A ??x?? cutout of the final reduced image
is shown in Fig.~??b and again in Fig.~??c with contours from the
MERLIN radio observations of \citet{Marlow} overlaid.  

\section{Lens Modelling}
To model B1555+375, we use the lens modelling code {\tt glafic}
\citep{Oguri}.  The inputs to the model are the observed image positions
and flux densities measured by the radio observations of \citet{Marlow}.
These authors present a singular isothermal ellipsoid (SIE) model that
provides the first fit to the image positions but
does not reproduce the strong flux ratio anomaly between components A and B.
(see Fig.6, Table 2 \& 3 in \citet{Marlow}). In K-band
image (Fig. 1, left), a recognizable edge-on disk is lie in the center
with position angle $~10$ degree. With this new information from A/O image, we model
B1555+375 with two models. The first model consists of an SIE bulge
and an exponential edge-on disk (SIE+Expdisk). The second model
consists of two SIE mass profiles (2-SIE), for bulge and disk
respectively. We assume that the lens galaxy has $z_{l}=0.5$ and the source has $z_s=1.5$. The full best-fit parameters are shown in Table 1. Both
of these models predict fairly matched results with the radio
observation. The comparison of observation and SIE+Expdisk result is
in Figure 2, Table 2 \& 3.
%These authors present a singular isothermal ellipsoid (SIE) model that
%provides an excellent fit to the image positions (?? IS THIS TRUE?--NO we get a better model result)



%\begin{figure}
%\plotone{B1555_Kp.eps}
%\plottwo{B1555_model.eps}{B1555_residue.eps}
%\caption{Left: SHARP K-band image of B1555+375 Middle: Lens model image generated by glafic Right: Residue image of B1555+375.\\
%An edge-on disk can be seen in the center of SHARP K-band image. Exponential disk is marked by the green ellipse in residue image. We believe strong dust extinction from the edge-on disk makes B and D no detection in infrared. Thus leaving two negative holes in our residue.\label{fig1}}
%\end{figure}
%
%\begin{figure}
%\plotone{point_source.eps}
%\caption{Radio observation(red open circle) and model-predicted(blue plus sign) image positions of B1555+375. The position of the source is at $(-0.2066,-0.1634)$ for SIE+Expdisk model, marked by a black filled circle. From left to right, four components are A, B (merging doubles), D(lowest spot), and C(right).\label{fig2}}
%\end{figure}

%%table1
\begin{table}
\begin{center}
\caption{Lens Model Best-Fit Parameters.\label{tbl-1}}
\begin{tabular}{ccccccc}
%%\tableline\tableline
		&SIE+Expdisk& 2-SIE		   
\\
\tableline\tableline
$x_1$  	& $-0.1883$	& $-0.1809$	  \\
$y_1$	&$-0.1923$	&$-0.1702$	  \\
$x_2$	&$-0.1615$ 	&$-0.1592$	  \\
$y_2$	&$-0.2502$	& $-0.1833$	  \\
$\sigma_1$	&$92.90$ &	$95.05$	  \\
$M_{tot} / \sigma_2$& $1.17\times 10^{10} $  &$81.36$ 	 \\  
$e_1$	& $0.27$	& $0.27$ \\  
$\theta_1$	&$105.5 \degr$ & $97.7 \degr$	 \\
$e_2$	&$0.84$	&$0.83$      \\
$\theta_2$	&$7.4\degr$ &$7.4\degr$  \\
$r_e$	& $0.20 ''$ &  -- \\
\tableline

\end{tabular}
%% Any table notes must follow the \end{tabular} command.
\tablecomments{Subscripts 1 and 2 represent bulge and disk components in each model respectively. Positions are offsets from radio image component A measured in units of arcsec. The velocity dispersion $\sigma$ is in units of $km/s$ and disk mass $M_{tot}$ is in units of $h^{-1} M_{\odot}$. $e$ is ellipticity and $\theta$ is the position angle measured east of north.}
\end{center}
\end{table}

%%table2
\begin{table}
\begin{center}
\caption{Radio Observation and Model-predicted Image Positions.\label{tbl-2}}
\begin{tabular}{lccccc}
\tableline\tableline
					&Radio	&		 & Model-predicted \\
Component &East &North &East 		&North\\ 
\tableline
A ........ &$0$    		&$0$		&$-0.0000$ &$+0.0000$   \\  
B ........ &$-0.0726$ 	&$+0.0480$	&$-0.0726$ &$+0.0479$   \\  
C ........ &$-0.4117$  &$-0.0280$	&$-0.4117$ &$-0.0280$   \\  
D ........ &$-0.1619$  &$-0.3680$	&$-0.1609$ &$-0.3678$   \\  
\tableline
\end{tabular}
%% Any table notes must follow the \end{tabular} command.
\tablecomments{Radio observation data quoted from Table 2 in \citet{Marlow}. Position offsets are in unit of arcsec.}
\end{center}
\end{table}

%%table3
\begin{table}
\begin{center}
\caption{Flux Ratios of B1555+375 Components.\label{tbl-3}}
\begin{tabular}{lccccc}
\tableline\tableline
				&5 GHz & 15 GHz  &Model-predicted\\
\tableline
A/B			&$1.8$ & $1.8$ & $1.7$  \\ 
A/C 		&$2.0$ 	&$2.4$ &$2.2$  \\
A/D			&$13.0$ &$ 12.9$ & $8.0$  \\
\tableline
\end{tabular}
%% Any table notes must follow the \end{tabular} command.
\tablecomments{Comparision between radio data and model-predicted flux ratios. Radio observation data quoted from Table 2 and Table 3 in \citet{Marlow}.}
\end{center}
\end{table}



We then use an extended source to model SHARP infrared image. In Figure 1, B and D have no detection in K-band, which believed due to strong dust extinction from the edge-on disk. The region and orientation of exponential disk in lens model are consistent with this hypothesis. The flux ratios in infrared are much higher than in radio also supports as an evidence of dust extinction.

\section{Discussion}
The abundance of substructure within galaxy-size halo has been an undetermined problem in structure formation studies. Discrepency between numerical simulation and observation within the local universe gives two possible interpretations: an alternative structure formation model or small sample from the Local Universe. Difficulties emerge while searching beyond the Local Group due to the low luminosity of satellites. This makes gravitational lensing a powerful tool for the detection of substructures. With the fact that surface brightness is sensitive to gravitational perturbation, flux ratio anomaly in strong lensing system becomes a sign of existence of subhalo and also reflects its total mass. Give directly measurement of mass function which can avoid making assumptions on how baryonic matter distributes with dark matter.

Discussions on substructure searching in strong lensing system are mostly done by multiple images of lensed quasar. In the ideal case (no extinction and micro lensing), flux ratio anomalies are brought by substructure perturbations with the assumption that lens galaxy has smooth gravitational potential. Numerical simulations have examined this idea with known lensed quasars. In \citet{Xu14}, they found that substructure cannot expalin for all of radio flux anomalous phenomena. Only 1 out of 6 lens systems can be reproduced the flux anomaly by adding subhalos on to the system. The rest of them are only few percent likely to have substructure origined flux anomaly, including our target B1555+375. They conclude B1555+375 with improper lens model. This result can be supported by SHARP K-band image, which an edge-on disk lies across between the radio merging double A \& B. A more complex lens model (e.g. include an exponential disk) must be applied on B1555+375.

In part 3, we present new lens models of B1555+375 with a bulge and an edge-on disk. Our models better match to the radio image positions than in \citet{Marlow} and also explain the flux ratio anomaly between components A and B. In our modelling, substructure is not neccessary to presence and a  exponential disk with high ellipcity well explains the observed flux ratio anomaly. This indicates that in B1555+375, flux ratio anomaly is caused by structure of lens galaxy itself rather than substructure. Although the number of radio loud flux anomalous lensed quasars remains small, new technique using narrow-line emission will enlarge the sample and bring statistical perspective on substructures in the near future. Our discover brings out a reminder: not all of flux ratio anomalies in lensed quasar indicate the existence of substructure even without the problem of dust extinction and micro lensing from stars. A complex lens galaxy mass distribution can dominate flux ratio anomalous phenomenon in strong lensed quasar. Thus, one must fully consider non-substructure effects before gathering any statistics or serious analysis. Otherwise, these ``false'' substructures that detected by flux anomalies can serverly biased the data analysis. For the case oringinated by real  substructures, mass distribution of lens galaxy may also influence the estimated mass and other derived properties. We would like to draw attention on this non-substructure effect in flux anomalous lensed quasar. For precise estimation on substurcture population, mass function, or any derived properties from strong lensed flux anomalous quasars, non-substructure effects are important to be understood beforehand.

Our result also shows that multi-wavelength image is helpful on exploring flux anomalous lensed quasars. Although radio band provides reliable data on position and fluxe of lensed images, optical and near-infrared imaging provides useful information about mass distribution of lens galaxy. In the previous paper of SHARP, \citet{SHARP12} show that high resolution IR imaging improves the lens modelling. The improvement is even more dramatic in B1555+375 while adding an edge-on disk can well explain flux ratio anomaly. Lack of knowledge on lens galaxy surely leads to an improper lens model sometimes and can have impacts on statics and derived properties among substructure detection. With the first system being analyzed and shown that an edge-on disk can explain its flux ratio anomaly, our follow up will be explore other SHARP systems. In Figure 3, all of these systems have a recognizable edge-on disk in infrared imaging. The flux ratio anomalies of these systems may also able to be explained by an edge-on disk structure.\\
 



\bibliographystyle{apj}
\bibliography{reference.bib}


%\begin{thebibliography}{}
%\bibitem[Oguri(2010)]{Oguri} Oguri, M. 2010, \pasj, 62, 1017
%\bibitem[Marlow et al.(1999)]{Marlow} Marlow, D. R., et al. 1999, \apj, 118, 654
%\end{thebibliography}

\end{document}

