\documentclass[manuscript]{emulateapj}

\shorttitle{Gravitational Lens Modeling of Flux Ratio Anomalous System B1555+375}
\shortauthors{TBD}

\begin{document}

\title{Gravitational Lens Modeling of Flux Ratio Anomalous System B1555+375}
\author{TBD}

\begin{abstract}
With the latest infrared image of B1555+375, an edge-on disk is shown to over cross the strong lensed images. Indicating the flux ratio anomalie of B1555+375 is dominanted by its edge-on disk structure. We present a lens model with SIE bulge and exponential disk, base on the radio observation from \citet{Marlow}. And apply this model to SHARP infrared image.

\end{abstract}

\keywords{ gravitaional lensing}

\section{Introduction}
Flux ratio anomaly in gravitational lensed systems has been seen as a promising method to probe substructures within halos.
Base on the concordant cold dark matter cosmology, simulations show more substructures than up-to-date observation searches.
This ``missing satellite'' problem arose interesting discrepancy between theoretical and observational understanding on structure
formation. \\
One possible explanation to this discrepancy is the difficulties of searching low-mass satellites(typical mass range?). Recent research shows using gravitational lensing effects from substructures can lower the detectable mass limit (mass range). More satellite galaxy candidates would be discovered from flux ratio anomaly in strong lensed galaxy systems, which may give us new insights to structures within dark matter halo. \\
On the other hand, simulations haven't reach a concise conclusion on the cause of flux ratio anomaly. Substructures may not be the only reason for flux ratio anomalies in merging double or triple images. Dust extinction, micro lensing from stars, and structures belongs to lens galaxy can also contribute to flux ratio anomaly. These factors would affect the estimation of substructure mass and even give false signal when searching possible satellite candidates.\\
In the latest SHARP observation, we aimed on several strong lensed systems with flux ratio anomaly discovered. Clear edge-on disk is found in the infrared images of (systems). To probe the causes of flux ratio anomaly, we model one system, B1555+375, with a center bulge and an edge-on disk. (B1555 radio observation details in intro?) In this paper, we present the latest SHARP data in part 2, lens model of B1555 in part 3, and discussion in part 4. 

\section{Data Collection \& Reduction}

\section{Lens Modeling}
There are several public lens modeling codes allow researchers to model strong lensed systems with most common mass models. We use glafic \citep{Oguri} to explore B1555 with known position and flux data from the radio observation done by \citet{Marlow}. Morlow et al. first published the flux ratio anomaly detection and lens model of B1555+375. However, their SIE lens model cannot explain the flux ratio anomaly between image A and B (see Fig.6, Table 2 \& 3 in \citet{Marlow}). In K-band image (Fig. 1, left), a recognizable edge-on disk is lie in the center with position angle $~10$ degree. With this information, we model B1555+375 with SIE for bulge and exponential disk model for edge-on disk. The full best-fit parameters are shown in Table 1.

The comparison of radio observation and our modeling result is in Figure 2, Table 2 \& 3.\\




\begin{figure}
\plotone{B1555_Kp.eps}
\plottwo{B1555_model.eps}{B1555_residue.eps}
\caption{Left: SHARP K-band image of B1555+375 Middle: Lens model image generated by glafic Right: Residue image of B1555+375.\\
Exponential disk is marked by the green ellipse in residue image. We believe strong dust extinction from the edge-on disk makes B and D no detection in infrared. Thus leaving two negative holes in our residue.\label{fig1}}
\end{figure}

\begin{figure}
\plotone{point_source.eps}
\caption{Radio observation(red open circle) and model-predicted(blue plus sign) image positions of B1555+375. The position of the source is at $(-0.2066,-0.1634)$, marked by a black filled circle. From left to right, four components are A, B (merging doubles), D(lowest spot), and C(right).\label{fig2}}
\end{figure}

%%table1
\begin{table}
\begin{center}
\caption{Lens Model Best-Fit Parameters.\label{tbl-1}}
\begin{tabular}{lcccccc}
%%\tableline\tableline
Model &		&		 &  \\
\tableline\tableline
SIE 	& $x$		& $y$ 		& $\sigma (km/s)$ 	& $e$	& $\theta$  \\  
		&$-0.1883$ 	&$-0.1923$	&$93.74$ 			&$0.27$	& $105.5 \degr$   \\
\tableline
Expdisk	& $x$		& $y$		& $M_{tot} (h^{-1}M_{\odot})$	& $e$	& $\theta$	& $r_e$ \\
		&$-0.1615$  &$-0.2502$	&$1.19\times 10^{10} $ 			&$0.84$ &$7.4 \degr$& $0.20 ''$  \\  
 
\tableline
\end{tabular}
%% Any table notes must follow the \end{tabular} command.
\tablecomments{Positions are offsets measured from component A in unit of arcsec. $e$ is ellipticity and $\theta$ is the position angle measured east of north.}
\end{center}
\end{table}

%%table2
\begin{table}
\begin{center}
\caption{Radio Observation and Model-predicted Image Positions.\label{tbl-2}}
\begin{tabular}{lccccc}
\tableline\tableline
					&Radio	&		 & Model-predicted \\
Component &East &North &East 		&North\\ 
\tableline
A ........ &$0$    		&$0$		&$-0.0000$ &$+0.0000$   \\  
B ........ &$-0.0726$ 	&$+0.0480$	&$-0.0726$ &$+0.0479$   \\  
C ........ &$-0.4117$  &$-0.0280$	&$-0.4117$ &$-0.0280$   \\  
D ........ &$-0.1619$  &$-0.3680$	&$-0.1609$ &$-0.3678$   \\  
\tableline
\end{tabular}
%% Any table notes must follow the \end{tabular} command.
\tablecomments{Radio observation data quoted from Table 2 in \citet{Marlow}. Position offsets are in unit of arcsec.}
\end{center}
\end{table}

%%table3
\begin{table}
\begin{center}
\caption{Flux Ratios of B1555+375 Components.\label{tbl-3}}
\begin{tabular}{lllccc}
\tableline\tableline
				&Radio &Infrared  &Model-predicted\\
\tableline
A/B			&$1.8$ &$3.5$ &$1.7$  \\ 
A/C 		&$2.0$ &$1.0$ &$2.2$  \\
A/D			&$13.0$ &$5.4$ &$8.0$  \\
\tableline
\end{tabular}
%% Any table notes must follow the \end{tabular} command.
\tablecomments{Radio observation data quoted from Table 2 in \citet{Marlow}. Flux ratio between the merging double is much higher in infrared than in radio, which is believed to be caused by strong dust extinction from the edge-on disk.}
\end{center}
\end{table}



We then use an extended source to model SHARP infrared image. In Figure 1, B and D have no detection in K-band, which believed due to strong dust extinction from the edge-on disk. The region and orientation of exponential disk in lens model are consistent with this hypothesis. The flux ratios in infrared are much higher than in radio also supports as an evidence of dust extinction.

\section{Discussion}
We have seen a similar flux ratio anomaly pattern in our edge-on disk lens model of B1555+375. The model predicted point source images also well matched with radio observation. Our modeling result indicates structures of lens galaxy itself can dominant flux ratio anomalies on lensing images. 
Our follow up will be explore more similar SHARP systems with flux ratio anomaly and a recognizable edge-on disk. As shown in Figure 3, the flux ratio anomalies of these systems may also able to be explained by an edge-on disk structure.\\
We still have great expectation on the substructure searching approach from flux ratio anomaly by using strong lensing systems. At the meanwhile, we would like to draw some attention on other factors can cause flux ratio anomaly. These factors also give a clue of the structure and mass distribution of lens galaxy, which can be important to research required really accurate measurements, like time-delay.  For substructure searching, one should reconsider and rule out the possibility of edge-on disk before doing any further analysis.\\
Our result also shows that multi-wavelength image is helpful on exploring flux ratio anomalous systems.

\bibliographystyle{apj}
\bibliography{reference.bib}


%\begin{thebibliography}{}
%\bibitem[Oguri(2010)]{Oguri} Oguri, M. 2010, \pasj, 62, 1017
%\bibitem[Marlow et al.(1999)]{Marlow} Marlow, D. R., et al. 1999, \apj, 118, 654
%\end{thebibliography}

\end{document}

